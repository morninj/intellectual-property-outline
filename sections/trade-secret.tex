\section{Trade Secret}

\subsection{Introduction}

\subsubsection{History}

\begin{enumerate}
    \item Acto servi corrupti: corrupting a slave, i.e., bribing to get secret 
    information.\footnote{Casebook p. 33.}
\end{enumerate}

\subsubsection{Overview}

\begin{enumerate}
    \item Traditionally a common law tort.\footnote{Casebook p. 35.}
    \item 1979: Uniform Trade Secrets Act (UTSA), adopted by 44 states and 
    DC.\footnote{Casebook p. 36.}
    \item Structure of a trade secret claim:\footnote{Casebook p. 37.
    }
    \begin{enumerate}
        \item Trade secret--eligible \textbf{subject matter}.
        \begin{enumerate}
            \item Information that derives value from being kept secret.
            \item Must be secret.
        \end{enumerate}
        \item \textbf{Reasonable precautions} to keep it secret.
        \item Misappropriation.
    \end{enumerate}
    \item Features of trade secret law:
    \begin{enumerate}
        \item Potentially unlimited protection---as long as the information is 
        valuable and secret.
        \item Cases often, but not always, arise from preexisting 
        relationships, like employment contracts or expectations of 
        confidentiality.
    \end{enumerate}
    \item Federal IP laws \textbf{do not preempt state trade secret laws}.
\end{enumerate}

\subsubsection{Theory}

\begin{enumerate}
    \item Utilitarian: protecting secrets against theft encourages investment 
    in proprietary information.\footnote{Casebook p. 37.}
    \item Tort theory: prevent and punish wrongful acts---``the maintenance of 
    commercial morality.''\footnote{Casebook p.  37--38.}
\end{enumerate}

\subsection{Subject Matter}

\subsubsection{Defining Trade Secrets: \emph{Metallurgical Industries, Inc. v. 
Fourtek, Inc.}}

To determine whether information is a trade secret, courts should consider the 
level of confidentiality, the value to the secret holder, and the cost of 
developing the secret.

\begin{enumerate}
    \item Metallurgical developed improvements to the zinc recovery process. 
    It alleged that these improvements were trade secrets and that former 
    employees stole them and brought them to Fourtek.\footnote{Casebook p. 40.}
    \item The district court held that the modification process was not 
    protectable as a trade secret.
    \item What is a trade secret?
    \begin{enumerate}
        \item Metallurgical's improvements were unknown to the industry. It 
        took steps to protect this information (e.g., NDAs), suggesting it was 
        secret.\footnote{Casebook p. 41.}
        \item Courts should \emph{consider} these factors, but they are not 
        \emph{required}:\footnote{Casebook p. 41--42.}
        \begin{enumerate}
            \item Confidentiality.
            \item Value to the secret holder.
            \item Cost of developing the secret.
        \end{enumerate}
    \end{enumerate}
    \item Here, Metallurgical had a trade secret.
\end{enumerate}

\subsubsection{Reasonable Efforts to Maintain Secrecy: \emph{Rockwell Graphic 
Systems, Inc. v. DEV Industries, Inc.}}

Evidence of reasonable precautions makes acquisition through ``proper means'' 
very unlikely.

\begin{enumerate}
    \item Rockwell manufactured printing presses. It sometimes subcontracted 
    the manufacture of replacement parts, which involved providing 
    subcontractors with ``piece part drawings'' of the parts to be 
    made.\footnote{Casebook p. 49.}
    \item Some of Rockwell's employees left for DEV, a competitor. They 
    allegedly copied some of the drawings.
    \item DEV argued that the drawings were not trade secrets because Rockwell 
    ``made only perfunctory efforts to keep them secret.''\footnote{Casebook 
    p. 50.} Rockwell did take some measures to preserve secrecy, but DEV 
    argues that it could have done more.
    \item Posner described two conceptions of requirement that the owner take 
    reasonable precautions to preserve secrecy:\footnote{Casebook p. 51.}
    \begin{enumerate}
        \item \emph{Deterrence}: prevent actions that redistribute wealth from 
        one firm to another. The plaintiff must prove that the defendant 
        obtained the secret through a wrongful act.
        \item \emph{Utilitarian}: encourage investment. The owner's actions 
        have evidentiary significance, but mainly to show that the secret has 
        value. It matters less how the defendant acquired the information.
    \end{enumerate}
    \item Under both conceptions, efforts to preserve secrecy also matter for 
    remedies, because if the owner had let the secret fall into the public 
    domain, he would receive a windfall by enforcing exclusive ownership of 
    the secret.\footnote{Casebook p. 52.}
    \item Here, whether Rockwell's efforts were ``reasonable'' is a question 
    of fact. Motion for summary judgment denied.
    \item Perfect security can be inefficient, so it is not 
    required---``perfect security is not optimum security.'' The standard is 
    reasonableness.
\end{enumerate}

\subsubsection{Disclosure of Trade Secrets: \emph{Data General Corp. v. 
Digital Computer}}

Public disclosure destroys the secret, but as long as the secret remains 
secret, it's protectable. A non-disclosure agreement \emph{may} be good 
enough.

Trade secrets can be disclosed in several ways: (1) publication, (2) sale of a 
product that embodies the secret, (3) disclosure by someone other than the 
owner, (4) inadvertent disclosure, or (5) forced disclosure by government 
agencies.\footnote{Casebook p. 60-63.}

\begin{enumerate}
    \item Data General included design drawings with the computers it sold. 
    The drawings included an NDA. Digital acquired the drawings from one of 
    Data General's customers and used it as a template to develop a competing 
    product.\footnote{Casebook p. 58--59.}
    \item Held: whether Data General took adequate measures to protect its 
    secret was a factual question.
\end{enumerate}

\subsection{Misappropriation of Trade Secrets}

\subsubsection{improper means: \emph{e.i. dupont de nemours \& co. v. 
christopher}}

where a person has taken reasonable precautions to preserve secrecy, taking 
its secret is improper. illegal conduct or a breach of a confidential 
relationship are not required.

\begin{enumerate}
    \item Christopher was hired to fly over the duPont plant, which was under 
    construction, to discover their secret process for making methanol.
    \item The court held that duPont had taken reasonble precautions to 
    preserve secrecy, so the flyover was improper.\footnote{Casebook p. 67.}
\end{enumerate}

\subsubsection{Confidential Relationship: \emph{Smith v. Dravo Corp.}}

Confidential relationships can be implied from the circumstances.

\begin{enumerate}
    \item Dravo expressed interest in buying Smith's shipping container 
    business. Smith sent detailed information to Dravo about its 
    operations.\footnote{Casebook p. 70--71.}
    \item Dravo used Smith's designs to build its own containers.
    \item Smith did not get an express promise to keep the information 
    confidential. But everybody understood that Smith was disclosing the 
    information solely for the purpose of appraising the business. ``There can 
    be no question that defendant knew and understood this limited purpose.''
\end{enumerate}

\subsubsection{Reverse Engineering: \emph{Kadant, Inc. v. Seeley Machine, Inc.}}

Reverse engineering is a legitimate use of another's trade secret.

\begin{enumerate}
    \item After Kadant terminated Corlew, Corlew went to work for Seeley, 
    which developed a similar product line.
    \item Kadant argued that the only way Seeley could have developed the 
    product so fast was through trade secret theft. Seeley argued that it 
    reverse engineered Kadant's products.
    \item Seeley argued that the products were simple, so reverse engineering 
    was quick and easy. Held: Kadant failed to rebut this argument.
\end{enumerate}

\subsubsection{The Special Case of Departing Employees}

\begin{enumerate}
    \item What's the optimal rate of employee mobility?
    \item Two competing needs: (1) need to train employees and (2) need to 
    trust employees.
    \item Tools to protect secrets in the context of departing employees: 
    non-compete agreements, IP assignment clauses, and trade secrecy per se.
    \item A lot of the action is in the evidence.
\end{enumerate}

\paragraph{IP Assignment Clauses}

\begin{enumerate}
    \item Typically broad (why not?).
    \item Some states (``right to invent'' states) regulate the effect of 
    broad IP assignment clauses.
    \item Common law of IP ownership---three categories of employee (which 
    only apply when there is not a formal contract---i.e., almost never):
    \begin{enumerate}
        \item ``Employed to invent.'' IP rights belong to the employer, even 
        in the absence of an explicit agreement.
        \item ``Inventions made with employer resources''---e.g., somebody who 
        supervises manufacture. Employer gets a ``shop right,'' meaning they 
        can use the invention, but they don't own it.
        \item ``Independent invention.'' Employee owns it.
    \end{enumerate}
    \item Limits on employment contracts:
    \begin{enumerate}
        \item E.g., Cal. Labor Code \S\ 2870: can't require assignment of 
        completely independent inventions.
        \item \emph{Roberts v. Sears Roebuck}: salesman who invented a 
        quick-release socket wrench on his own time did not have an obligation 
        to assign.
    \end{enumerate}
    \item Trailer clauses (or follow-on clauses): IP assignment agreement 
    extends X months after the period of employment.
    \begin{enumerate}
        \item \emph{General Signal}: employee waited until days after the 
        expiration of the trailer clause. Held: the invention was actually 
        conceived during the trailer period; the employee's timing was too 
        convenient.
    \end{enumerate}
\end{enumerate}

\paragraph{Noncompete Agreement: \emph{Edwards v. Arthur Andersen}}
~\\\\
California invalidates noncompete clauses in employment contracts, unless they 
are necessary to protect the employer's trade secrets. The policy is to 
promote ``free and full practice of one's profession.'' But California is a 
distinct minority. Most states enforce noncompetes.

\begin{enumerate}
    \item Cal. Bus. \& Prof. Code \S\ 16600: noncompetes are per se 
    unenforceable in CA, but there are many exceptions---e.g., you can require 
    nondisclosure of confidential information.
    \begin{enumerate}
        \item Most states would impose a common law reasonableness 
        requirement, but CA does not.
    \end{enumerate}
    \item Agreement in this case: no similar work for eighteen months, and no 
    customer poaching for twelve months.\footnote{Casebook p. 90.}
    \item HSBC acquired Arthur Andersen and required employees to sign 
    noncompetes. Edwards was fired for refusing to sign. He argued that the 
    agreement was unenforceable.
    \item The court agreed with Edwards. California law protects the freedom 
    to pursue enterprise and employment.\footnote{Casebook p. 92.}
    \item Classic exception (footnote 4): nondisclosure agreements are 
    enforceable if they are necessary to protect trade secrets.
    \item Choice of law: even if the noncompete is signed in another state, 
    and indicates that that state's laws applies, CA courts will \emph{still} 
    likely apply their laws and find the agreements to be unenforceable.
\end{enumerate}

\paragraph{``Inevitable Disclosure''}

\begin{enumerate}
    \item Pepsi won an injunction to prevent an executive from going to 
    Quaker, because disclosure of trade secrets would have been inevitable. A 
    confidentiality agreement wasn't enough.\footnote{Casebook p. 99--101.}
    \item But CA appellate courts have rejected the inevitable disclosure 
    doctrine as counter to the policy of employee mobility.
\end{enumerate}

\subsection{Agreements to Keep Secrets}

\begin{enumerate}
    \item If the information doesn't meet the requirements to be a trade 
    secret, can the parties still agree by contract to keep it 
    secret?\footnote{Casebook p. 106.}
    \item \emph{Warner-Lambert}: yes. But many courts disagree.
\end{enumerate}

\subsubsection{\emph{Warner-Lambert Pharmaceutical Co. v. John J. Reynolds}}

Destruction of the trade secret (e.g., by public disclosure) does not always 
render license agreement unenforceable.

\begin{enumerate}
    \item Warner-Lambert licensed the secret formula for Listerine. After the 
    formula became public, it sued for declaratory judgment to remove its 
    obligation to pay royalties.\footnote{Casebook p. 107.}
    \item Held: the agreement was indefinite. ``One who acquires a trade 
    secret or secret formula takes it subject to the risk that there be a 
    disclosure.\footnote{Casebook p. 108--09.}
    \item Many courts disagree.\footnote{Casebook p. 109--11.}
\end{enumerate}

\subsection{Remedies}

\begin{enumerate}
    \item The UTSA allows injunctions, damages, and attorney's 
    fees.\footnote{Casebook p. 111--12.}
    \item Injunctive relief casts trade secrets as property, but damages sound 
    like restitution, which would make the aim trade secret remedies to make 
    the plaintiff whole, as in tort or contract.
    \item Criminal provisions: CA penal code \S\ 499(c); federal Economic 
    Espionage Act (1996), \S\ 1831.
    \begin{enumerate}
        \item Elements of an EEA offense:
        \begin{enumerate}
            \item Defendant appropriated proprietary information.
            \item Knew that it was proprietary.
            \item The information was a trade secret.
            \item Intended to economically benefit someone other than the 
            trade secret owner.
            \item Trade secret is related to a product involved in interstate 
            or foreign commerce.
        \end{enumerate}
        \item Penalties: fines, jail time.
    \end{enumerate}
\end{enumerate}

\subsubsection{Head-Start Injunction: \emph{Winston Research Corp. v. 3M Corp.}}

\begin{enumerate}
    \item Mincom employees developed tape recorder technology and then left to 
    form Winston to market the same technology.
    \item Remedies aim to balance the rights of the employer and the 
    employee.\footnote{Casebook p. 113--14.}
    \item There was trade secret misappropriation here.
    \item The district court granted (and the court here affirmed) a two-year 
    injunction. A permanent injunction would harm employee mobility and the 
    public interest, while no injunction would give the culpable employees an 
    unfair head start in the market. A limited injunction would be a good 
    compromise.
\end{enumerate}
