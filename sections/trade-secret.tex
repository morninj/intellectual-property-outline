\section{Trade Secret}

\subsection{Introduction}

\subsubsection{History}

\begin{enumerate}
    \item Acto servi corrupti: corrupting a slave, i.e., bribing to get secret 
    information.\footnote{Casebook p. 33.}
\end{enumerate}

\subsection{Overview}

\begin{enumerate}
    \item Traditionally a common law tort.\footnote{Casebook p. 35.}
    \item 1979: Uniform Trade Secrets Act (UTSA), adopted by 44 states and 
    DC.\footnote{Casebook p. 36.}
    \item Structure of a trade secret claim:\footnote{Casebook p. 37.
    }
    \begin{enumerate}
        \item Subject matter.
        \item Reasonable precautions to keep it secret.
        \item Misappropriation.
    \end{enumerate}
\end{enumerate}

\subsection{Theory}

\begin{enumerate}
    \item Utilitarian: protecting secrets against theft encourages investment 
    in proprietary information.\footnote{Casebook p. 37.}
    \item Tort theory: prevent and punish wrongful acts---``the maintenance of 
    commercial morality.''\footnote{Casebook p.  37--38.}
\end{enumerate}

\subsection{Subject Matter}

\subsubsection{Defining Trade Secrets: \emph{Metallurgical Industries, Inc. v. 
Fourtek, Inc.}}

To determine whether information is a trade secret, courts should consider the 
level of confidentiality, the value to the secret holder, and the cost of 
developing the secret.

\begin{enumerate}
    \item Metallurgical developed improvements to the zinc recovery process. 
    It alleged that these improvements were trade secrets and that former 
    employees stole them and brought them to Fourtek.\footnote{Casebook p. 40.}
    \item The district court held that the modification process was not 
    protectable as a trade secret.
    \item What is a trade secret?
    \begin{enumerate}
        \item Metallurgical's improvements were unknown to the industry. It 
        took steps to protect this information (e.g., NDAs), suggesting it was 
        secret.\footnote{Casebook p. 41.}
        \item Courts should \emph{consider} these factors, but they are not 
        \emph{required}:\footnote{Casebook p. 41--42.}
        \begin{enumerate}
            \item Confidentiality.
            \item Value to the secret holder.
            \item Cost of developing the secret.
        \end{enumerate}
    \end{enumerate}
    \item Here, Metallurgical had a trade secret.
\end{enumerate}
% TODO notes 43-48

\subsubsection{Reasonable Efforts to Maintain Secrecy: \emph{Rockwell Graphic 
Systems, Inc. v. DEV Industries, Inc.}}

% TODO add takeaway

\begin{enumerate}
    \item Rockwell manufactured printing presses. It sometimes subcontracted 
    the manufacture of replacement parts, which involved providing 
    subcontractors with ``piece part drawings'' of the parts to be 
    made.\footnote{Casebook p. 49.}
    \item Some of Rockwell's employees left for DEV, a competitor. They 
    allegedly copied some of the drawings.
    \item DEV argued that the drawings were not trade secrets because Rockwell 
    ``made only perfunctory efforts to keep them secret.''\footnote{Casebook 
    p. 50.} Rockwell did take some measures to preserve secrecy, but DEV 
    argues that it could have done more.
    \item Posner described two conceptions of requirement that the owner take 
    reasonable precautions to preserve secrecy:\footnote{Casebook p. 51.}
    \begin{enumerate}
        \item \emph{Deterrence}: prevent actions that redistribute wealth from 
        one firm to another. The plaintiff must prove that the defendant 
        obtained the secret through a wrongful act.
        \item \emph{Utilitarian}: encourage investment. The owner's actions 
        have evidentiary significance, but mainly to show that the secret has 
        value. It matters less how the defendant acquired the information.
    \end{enumerate}
    \item Under both conceptions, efforts to preserve secrecy also matter for 
    remedies, because if the owner had let the secret fall into the public 
    domain, he would receive a windfall by enforcing exclusive ownership of 
    the secret.\footnote{Casebook p. 52.}
    \item Here, whether Rockwell's efforts were ``reasonable'' is a question 
    of fact. Motion for summary judgment denied.
    \item Perfect security can be inefficient, so it is not 
    required---``perfect security is not optimum security.'' The standard is 
    reasonableness.
\end{enumerate}

% TODO remove
\newpage

\subsubsection{Disclosure of Trade Secrets}

% TODO 58-65

\subsection{Misappropriation of Trade Secrets}

\subsubsection{Improper Means}

% TODO 66-70

\subsubsection{Confidential Relationship}

% TODO 70-76

\subsubsection{Reverse Engineering}

% TODO 76-83

\subsubsection{The Special Case of Departing Employees}

% TODO 83-106; 11/5: expand

\begin{enumerate}
    \item What's the optimal rate of employee mobility?
    \item Two competing needs: (1) need to train employees and (2) need to 
    trust employees.
    \item Tools to protect secrets in the context of departing employees: 
    non-compete agreements, IP assignment clauses, and trade secrecy per se.
    \item A lot of the action is in the evidence.
\end{enumerate}

% Kadan TODO p. 79

\begin{enumerate}
    \item Did the defendant company legitimately reverse engineer the 
    technology, or did it misappropriate trade secrets via the former 
    employee (Corlew)?
    \item Held: reverse engineering was easy here, according to the 
    defendants, and the plaintiffs didn't successfully show the opposite.  
    \item 
\end{enumerate}

\paragraph{IP Assignment Clauses}

\begin{enumerate}
    \item Typically broad (why not?).
    \item Some states (``right to invent'' states) regulate the effect of 
    broad IP assignment clauses.
    \item Common law of IP ownership---three categories of employee (which 
    only apply when there is not a formal contract---i.e., almost never):
    \begin{enumerate}
        \item ``Employed to invent.'' IP rights belong to the employer, even 
        in the absence of an explicit agreement.
        \item ``Inventions made with employer resources''---e.g., somebody who 
        supervises manufacture. Employer gets a ``shop right,'' meaning they 
        can use the invention, but they don't own it.
        \item ``Independent invention.'' Employee owns it.
    \end{enumerate}
    \item Limits on employment contracts:
    \begin{enumerate}
        \item E.g., Cal. Labor Code \S\ 2870: can't require assignment of 
        completely independent inventions.
        \item \emph{Roberts v. Sears Roebuck}: salesman who invented a 
        quick-release socket wrench on his own time did not have an obligation 
        to assign.
    \end{enumerate}
    \item Trailer clauses (or follow-on clauses): IP assignment agreement 
    extends X months after the period of employment.
    \begin{enumerate}
        \item \emph{General Signal}: employee waited until days after the 
        expiration of the trailer clause. Held: the invention was actually 
        conceived during the trailer period; the employee's timing was too 
        convenient.
    \end{enumerate}
\end{enumerate}

\paragraph{Noncompete Agreement: \emph{Edwards v. Arthur Anderson}}

% TODO add takeaway: free and full practice of one's profession. (this is the 
% CA view; distinctly a minority view. most states enforce noncompetes)

\begin{enumerate}
    \item Cal. Bus. \& Prof. Code \S\ 16600: noncompetes are per se 
    unenforceable in CA, but there are many exceptions---e.g., you can require 
    nondisclosure of confidential information.
    \begin{enumerate}
        \item Most states would impose a common law reasonableness 
        requirement, but CA does not.
    \end{enumerate}
    \item Agreement in this case: no similar work for eighteen months, and no 
    customer poaching for twelve months.\footnote{Casebook p. 90.}
    \item HSBC acquired Arthur Anderson and required employees to sign 
    noncompetes. Edwards was fired for refusing to sign. He argued that the 
    agreement was unenforceable.
    \item The court agreed with Edwards. California law protects the freedom 
    to pursue enterprise and employment.\footnote{Casebook p. 92.}
    \item Classic exception (footnote 4): nondisclosure agreements are 
    enforceable if they are necessary to protect trade secrets.
    \item Choice of law: even if the noncompete is signed in another state, 
    and indicates that that state's laws applies, CA courts will \emph{still} 
    likely apply their laws and find the agreements to be unenforceable.
\end{enumerate}

\newpage % TODO remove

\subsection{Agreements to Keep Secrets}

\begin{enumerate}
    \item If the information doesn't meet the requirements to be a trade 
    secret, can the parties still agree by contract to keep it 
    secret?\footnote{Casebook p. 106.}
\end{enumerate}

\subsubsection{\emph{Warner-Lambert Pharmaceutical Co. v. John J. Reynolds}}

\begin{enumerate}
    \item Warner-Lambert licensed the secret formula for Listerine. After the 
    formula became public, it sued for declaratory judgment to remove its 
    obligation to pay royalties.\footnote{Casebook p. 107.}
    \item Held: the agreement was indefinite. ``One who acquires a trade 
    secret or secret formula takes it subject to the risk that there be a 
    disclosure.\footnote{Casebook p. 108--09.}
    \item Many courts disagree.\footnote{Casebook p. 109--11.}
\end{enumerate}

\subsection{Remedies}

\begin{enumerate}
    \item The UTSA allows injunctions, damages, and attorney's 
    fees.\footnote{Casebook p. 111--12.}
    \item Injunctive relief casts trade secrets as property, but damages sound 
    like restitution, which would make the aim trade secret remedies to make 
    the plaintiff whole, as in tort or contract.
\end{enumerate}

\subsubsection{Head-Start Injunction: \emph{Winston Research Corp. v. 3M Corp.}}

\begin{enumerate}
    \item Mincom employees developed tape recorder technology and then left to 
    form Winston to market the same technology.
    \item Remedies aim to balance the rights of the employer and the 
    employee.\footnote{Casebook p. 113--14.}
    \item There was trade secret misappropriation here.
    \item The district court granted (and the court here affirmed) a two-year 
    injunction. A permanent injunction would harm employee mobility and the 
    public interest, while no injunction would give the culpable employees an 
    unfair head start in the market. A limited injunction would be a good 
    compromise.
\end{enumerate}
