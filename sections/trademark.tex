\section{Trademark}

\subsection{Introduction}

\subsubsection{Background}

\begin{enumerate}
    \item Trademarks ``help to reduce information and transaction costs by 
    allowing customers to estimate the nature and quality of goods before 
    purchase.''\footnote{Casebook p. 763.}
    \item 1870: first federal trademark statute. Struck down as beyond 
    Congress's powers under the patent and copyright clause.\footnote{Casebook 
    p. 764.}
    \item 1881: second federal statute, passed under the Commerce Clause. 
    Significantly modified in 1905 and 1920.
    \item 1946: Lanham Act, 15 U.S.C. \S\S\ 1051 et seq.
    \item Trademark protections have generally expanded.\footnote{Casebook p. 
    765.}
\end{enumerate}

\subsubsection{A Brief Overview of Trademark Theory}

\begin{enumerate}
    \item Trademarks do not protect or reward novelty or invention. They 
    reward the first user of a mark.\footnote{Casebook p. 765.}
    \item Traditional trademark principles resemble tort law: preventing 
    unfair competition and consumer deception.
    \item Infringement actions can protect consumers.\footnote{Casebook p. 
    766.}
    \item Trademarks protect three kinds of investment:
    \begin{enumerate}
        \item Creation of the mark.
        \item Advertising.
        \item Product creation.
    \end{enumerate}
\end{enumerate}

\subsubsection{The Basic Economics of Trademark and Advertising}

\begin{enumerate}
    \item No consensus on the economic function of 
    trademarks.\footnote{Casebook p. 766.}
    \item Does advertising help or manipulate consumers? Does it communicate 
    valuable price or quality information, or does it artificially create 
    demand for nonessential features?
    \item Early commentators--``product differentiation theory'':
    \begin{enumerate}
        \item Trademarks are bad.
        \item Advertising unnaturally stimulates demand.
        \item Advertising perpetuates oligopoly. For instance, brand-name 
        drugs can sell for twice as much as chemically identical generic 
        drugs. Arguably, this hurts consumers.\footnote{Casebook pp. 766--67.}
    \end{enumerate}
    \item Now--``product information theory'':
    \begin{enumerate}
        \item Trademarks are good.
        \item Consensus that advertising cheaply conveys information to 
        consumers.
        \item Product's ``search characteristics'': price, color, shape, etc.
        \item Product's ``experience characteristics'': taste, long-term 
        durability. These are only apparent after purchase. Advertising and 
        trademarks can help identify these characteristics and encourage 
        repeat purchases.
    \end{enumerate}
    \item \emph{Consumer protection theory}: advertising cheaply conveys 
    valuable information to consumers.
    \item \emph{Producer incentive theory}: trademarks are ``essential 
    shorthand'' for consumers' positive associations with a 
    product.\footnote{Casebook p. 768.}
    \item Copyright protections have expanded beyond protecting against 
    consumer deception to a broader range of property-like 
    protections.\footnote{Casebook pp. 769--70.}
\end{enumerate}

\subsection{What Can Be Protected as a Trademark?}

\subsubsection{Trademarks, Trade Names, and Service Marks}

\begin{enumerate}
    \item \textbf{Trademark}: ``word, name, symbol, or device'' to identify 
    goods.\footnote{15 U.S.C. \S\ 1127; casebook p. 771.}
    \item \textbf{Service mark}: ``word, name, symbol, or device'' to identify 
    \emph{services}. Generally subject to the same rules as trademarks.
    \item \textbf{Trade names}: can only be registered if they \emph{identify 
    the source of particular goods}, rather than a company 
    alone.\footnote{Casebook p. 772.}
\end{enumerate}

\subsubsection{Certification and Collective Marks}

\begin{enumerate}
    \item \textbf{Certification mark}: ``word, name, symbol, or device'' to 
    certify characteristics of a product---i.e., a seal of approval. Used by 
    trade associations and commercial groups---e.g., the city of 
    Roquefort.\footnote{Casebook pp. 772--73.}
    \item \textbf{Collective mark}: trademark or service mark adopted by a 
    collective. It can be (1) used by its members to distinguish its products 
    from non-member products, or (2) indicating membership in a collective 
    group, like a union.\footnote{Casebook p. 773.} Marks of the first type 
    are useful in franchising arrangements.\footnote{Casebook p. 774.}
\end{enumerate}

\subsubsection{Trade Dress and Product Configuration}

\begin{enumerate}
    \item Design and packaging---and sometimes the design of the product 
    itself.\footnote{Casebook p. 774.}
\end{enumerate}

\subsubsection{Color, Fragrance, and Sounds: \emph{Qualitex Co. v. Jacobson 
Products Co., Inc.}}

\begin{enumerate}
    \item 
\end{enumerate}

\subsection{Establishment of Trademark Rights}

\subsubsection{Distinctiveness}

% TODO 781-828

\subsubsection{Priority}

% TODO 828-48

\subsubsection{Trademark Office Procedures}

% TODO 848-861

\subsubsection{Incontestability}

% TODO 861-68

\subsection{Infringement}

\subsubsection{Use as a Trademark}

% TODO 868-76

\subsubsection{Likelihood of Consumer Confusion}

% TODO 876-83

\subsubsection{Dilution}

% TODO 889-906

\subsubsection{Extension by Contract: Franchising and Merchandising}

% TODO 906-11

\subsubsection{Domain Names and Cybersquatting}

% TODO 911-930

\subsection{Defenses}

\subsubsection{Abandonment}

% TODO 953-66

\subsubsection{Nontrademark (or Nominative) Use, Parody, and the First 
Amendment}

% TODO 972-84

\subsection{Remedies}

\subsubsection{Injunctions}

% TODO 989

\subsubsection{Damages}

% TODO 989-1003
