\section{Trademark}

\subsection{Introduction}

\subsubsection{Background}

\begin{enumerate}
    \item Trademarks ``help to reduce information and transaction costs by 
    allowing customers to estimate the nature and quality of goods before 
    purchase.''\footnote{Casebook p. 763.}
    \item 1870: first federal trademark statute. Struck down as beyond 
    Congress's powers under the patent and copyright clause.\footnote{Casebook 
    p. 764.}
    \item 1881: second federal statute, passed under the Commerce Clause. 
    Significantly modified in 1905 and 1920.
    \item 1946: Lanham Act, 15 U.S.C. \S\S\ 1051 et seq.
    \item Trademark protections have generally expanded.\footnote{Casebook p. 
    765.}
\end{enumerate}

\subsubsection{A Brief Overview of Trademark Theory}

\begin{enumerate}
    \item Trademarks do not protect or reward novelty or invention. They 
    reward the first user of a mark.\footnote{Casebook p. 765.}
    \item Traditional trademark principles resemble tort law: preventing 
    unfair competition and consumer deception.
    \item Infringement actions can protect consumers.\footnote{Casebook p. 
    766.}
    \item Trademarks protect three kinds of investment:
    \begin{enumerate}
        \item Creation of the mark.
        \item Advertising.
        \item Product creation.
    \end{enumerate}
\end{enumerate}

\subsubsection{The Basic Economics of Trademark and Advertising}

\begin{enumerate}
    \item No consensus on the economic function of 
    trademarks.\footnote{Casebook p. 766.}
    \item Does advertising help or manipulate consumers? Does it communicate 
    valuable price or quality information, or does it artificially create 
    demand for nonessential features?
    \item Early commentators--``product differentiation theory'':
    \begin{enumerate}
        \item Trademarks are bad.
        \item Advertising unnaturally stimulates demand.
        \item Advertising perpetuates oligopoly. For instance, brand-name 
        drugs can sell for twice as much as chemically identical generic 
        drugs. Arguably, this hurts consumers.\footnote{Casebook pp. 766--67.}
    \end{enumerate}
    \item Now--``product information theory'':
    \begin{enumerate}
        \item Trademarks are good.
        \item Consensus that advertising cheaply conveys information to 
        consumers.
        \item Product's ``search characteristics'': price, color, shape, etc.
        \item Product's ``experience characteristics'': taste, long-term 
        durability. These are only apparent after purchase. Advertising and 
        trademarks can help identify these characteristics and encourage 
        repeat purchases.
    \end{enumerate}
    \item \emph{Consumer protection theory}: advertising cheaply conveys 
    valuable information to consumers.
    \item \emph{Producer incentive theory}: trademarks are ``essential 
    shorthand'' for consumers' positive associations with a 
    product.\footnote{Casebook p. 768.}
    \item Copyright protections have expanded beyond protecting against 
    consumer deception to a broader range of property-like 
    protections.\footnote{Casebook pp. 769--70.}
\end{enumerate}

\subsection{What Can Be Protected as a Trademark?}

\subsubsection{Trademarks, Trade Names, and Service Marks}

\begin{enumerate}
    \item \textbf{Trademark}: ``word, name, symbol, or device'' to identify 
    goods.\footnote{15 U.S.C. \S\ 1127; casebook p. 771.}
    \item \textbf{Service mark}: ``word, name, symbol, or device'' to identify 
    \emph{services}. Generally subject to the same rules as trademarks.
    \item \textbf{Trade names}: can only be registered if they \emph{identify 
    the source of particular goods}, rather than a company 
    alone.\footnote{Casebook p. 772.}
\end{enumerate}

\subsubsection{Certification and Collective Marks}

\begin{enumerate}
    \item \textbf{Certification mark}: ``word, name, symbol, or device'' to 
    certify characteristics of a product---i.e., a seal of approval. Used by 
    trade associations and commercial groups---e.g., the city of 
    Roquefort.\footnote{Casebook pp. 772--73.}
    \item \textbf{Collective mark}: trademark or service mark adopted by a 
    collective. It can be (1) used by its members to distinguish its products 
    from non-member products, or (2) indicating membership in a collective 
    group, like a union.\footnote{Casebook p. 773.} Marks of the first type 
    are useful in franchising arrangements.\footnote{Casebook p. 774.}
\end{enumerate}

\subsubsection{Trade Dress and Product Configuration}

\begin{enumerate}
    \item Design and packaging---and sometimes the design of the product 
    itself.\footnote{Casebook p. 774.}
\end{enumerate}

\subsubsection{Color, Fragrance, and Sounds: \emph{Qualitex Co. v. Jacobson 
Products Co., Inc.}}
~\\\\
Colors can be trademarked.

\begin{enumerate}
    \item Qualitex manufactured drycleaning pads with a distinctive green-gold 
    color.
    \item Circuit courts were split on whether colors could be trademarked.
    \item ``Symbol'' or ``device'' can refer to ``almost anything at all that 
    is capable of carrying meaning~.~.~.~''\footnote{Casebook p. 775.}
    \item Colors can signify brands. When a color signifies a product's 
    origin, it has taken on a ``secondary meaning.''\footnote{Casebook p. 776.}
    \begin{enumerate}
        \item \textbf{\enquote{\enquote{secondary meaning} is acquired when 
        \enquote{in the minds of the public, the primary significance of a 
        product feature~.~.~.~is to identify the source of the product rather 
        than the product itself}}}\footnote{Casebook p. 776.}
    \end{enumerate}
    \item Color meets the basic requirements for use as a trademark. It 
    can distinguish goods and identify their source without serving any other 
    function.\footnote{Casebook p. 777.}
    \item Jacobson made four arguments for why color alone should not be 
    granted trademark protections:
    \begin{enumerate}
        \item \emph{First}: it will create uncertainty (``shade confusion'').
        \begin{enumerate}
            \item But courts are skilled at making difficult decisions.
        \end{enumerate}
        \item \emph{Second}: colors are in limited supply.
        \begin{enumerate}
            \item Maybe, but that ``occasional problem'' does not justify a 
            ``blanket prohibition.'' 
            \item The \textbf{functionality doctrine} ``forbids the use of a 
            product's feature as a trademark where doing so will put a 
            competitor at a significant disadvantage because the feature is 
            \enquote{essential to the use or purpose of the article} or 
            \enquote{affects [its] cost or quality.}''\footnote{Casebook p. 
            779.} Here, if a limited supply of colors would harm competitors, 
            courts would not allow exclusive use of one of the colors as a 
            trademark.
        \end{enumerate}
        \item \emph{Third}: many older cases support its position.
        \begin{enumerate}
            \item No---those were pre--Lanham Act.
        \end{enumerate}
        \item \emph{Fourth}: firms can already use color as part of another 
        trademark or trade dress.
        \begin{enumerate}
            \item A company might have a reason to use a color instead of a 
            word or symbol.
        \end{enumerate}
    \end{enumerate}
\end{enumerate}

\subsection{Establishment of Trademark Rights}

\subsubsection{Distinctiveness}

\paragraph{Classification of Marks and Requirements for Protection}

\begin{enumerate}
    \item When a trademark can identify a \textbf{unique} product source, 
    rights are determined by priority of use.\footnote{Casebook p. 781.} These 
    types of marks are then further subdivided:
    \begin{enumerate}
        \item Arbitrary (Kodak, Ivory [for soap], Exxon).
        \item Fanciful (similar to arbitrary).
        \begin{enumerate}
            \item TODO difference between arbitrary and fanciful? 
            \emph{Zatarain's} conflates them---p. 784.
        \end{enumerate}
    \item Suggestive (Coppertone).
    \end{enumerate}
    \item All other marks---\textbf{generic trademarks}---require 
    \textbf{secondary meaning}. Types:
    \begin{enumerate}
        \item \textbf{Generic}: aspirin, cellophane. (Unprotectable even if 
        they have acquired secondary meaning.\footnote{Casebook p. 785.})
        \item \textbf{Descriptive}: describes something about the 
        goods or services---e.g., ``Tender Vittles'' for cat food or 
        ``Arthriticare'' for arthritis treatment.\footnote{Casebook p. 781.}
        \item \textbf{Geographic}.
        \item \textbf{Personal names}.
        \item ``Secondary meaning exists when buyers associate a descriptive 
        term with a single source of products.''\footnote{782.} Buyers do not 
        need to know the \emph{identity} of the source; they only need to know 
        that the product comes from a \emph{single} source.
    \end{enumerate}
\end{enumerate}

\paragraph{Descriptive Marks and Fair Use: \emph{Zatarain's, Inc. v. Oak Grove 
Smokehouse, Inc.}}
~\\\\
Descriptive terms can be trademarked, but the fair use defense allows 
competitors to use them in their original, descriptive sense. So, Zatarain's 
can have an exclusive right in the phrase ``Fish-Fri'' for its fish frying 
batter, but its competitors can use the words ``fish'' and ``fry'' to describe 
their own products.

\begin{enumerate}
    \item Zatarain's brought infringement suits for two of its registered 
    trademarks: ``Fish-Fri'' and ``Chick-Fri.''
    \item \textbf{Fair use} ``prevents a trademark registrant from 
    appropriating a descriptive term for its own use to the exclusion of 
    others, who may be prevent thereby from accurately describing their own 
    goods.''\footnote{Casebook p. 785.}
    \item ``Fish-Fri'':
    \begin{enumerate}
        \item Is ``Fish-Fri'' descriptive?
        \begin{enumerate}
            \item \emph{Dictionary}: refers to fried fish, so yes, 
            ``Fish-Fri'' is descriptive of the product (which is used to fry 
            fish).
            \item \emph{``Imagination test}'': if imagination is required to 
            associate the term with the product, the term is suggestive, not 
            descriptive. But here, no imagination is required to associate the 
            two.
            \item \emph{Competitive need}: do competitors need to use the term 
            to describe their products? Here, probably yes, because there is a 
            ``paucity of synonyms'' for ``fish'' and 
            ``fry.''\footnote{Casebook p. 787.}
            \item \emph{Actual use}: have competitors used the term to 
            describe their own products? Here, yes.
            \item So, ``Fish-Fri'' is descriptive.
        \end{enumerate}
        \item Does ``Fish-Fri'' have a secondary meaning?
        \begin{enumerate}
            \item Yes---survey evidence (of Louisiana, at least) and 
            circumstantial evidence showed that consumers associated 
            ``Fish-Fri'' with a specific source.
        \end{enumerate}
        \item Was there fair use?
        \begin{enumerate}
            \item Zatarain's cannot claim an exclusive right in the original, 
            descriptive sense of ``fish fry.'' So, its competitors are free to 
            use them in that sense.\footnote{Casebook p. 789.}
            \item Also, dissimilar trade dress would likely prevent consumer 
            confusion.
        \end{enumerate}
    \end{enumerate}
    \item ``Chick-Fri'':
    \begin{enumerate}
        \item Descriptive? Yes.\footnote{Casebook p. 789.}
        \item Secondary meaning? No (``paltry''---ha).\footnote{Casebook p. 
        790.}
    \end{enumerate}
\end{enumerate}

\newpage % TODO remove

\paragraph{Genericness}

% TODO 794

\paragraph{\emph{The Murphy Door Bed Co., Inc. v. Interior Sleep Systems, Inc.}}

% TODO 794-801

\paragraph{Genericide, Language, and Policing Costs}

% TODO 801-804

\paragraph{Distinctiveness of Trade Dress and Product Configuration}

% TODO 804

\paragraph{\emph{Two Pesos, Inc. v. Taco Cabana, Inc.}}

% TODO 804-810

\paragraph{\emph{Wal-Mart Stores, Inc. v. Samara Brothers, Inc.}}

% TODO 810-817

\paragraph{Functionality}

% TODO 817

\paragraph{\emph{TrafFix Devices, Inc. v. Marketing Displays, Inc.}}

% TODO 817-828

\subsubsection{Priority}

% TODO 828-48

\subsubsection{Trademark Office Procedures}

% TODO 848-861

\subsubsection{Incontestability}

% TODO 861-68

\subsection{Infringement}

\subsubsection{Use as a Trademark}

% TODO 868-76

\subsubsection{Likelihood of Consumer Confusion}

% TODO 876-83

\subsubsection{Dilution}

% TODO 889-906

\subsubsection{Extension by Contract: Franchising and Merchandising}

% TODO 906-11

\subsubsection{Domain Names and Cybersquatting}

% TODO 911-930

\subsection{Defenses}

\subsubsection{Abandonment}

% TODO 953-66

\subsubsection{Nontrademark (or Nominative) Use, Parody, and the First 
Amendment}

% TODO 972-84

\subsection{Remedies}

\subsubsection{Injunctions}

% TODO 989

\subsubsection{Damages}

% TODO 989-1003
