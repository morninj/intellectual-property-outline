\section{Patent}

\subsection{Introduction}

% TODO 123-31

\subsection{Elements of Patentability}

\subsubsection{Patentable Subject Matter}

% TODO 35 usc 101

% TODO 132-139 [compositions of matter: chakrabarty]

% TODO 158-177 [abstract ideas: bilski]

% TODO myriad, supp 3-11

\subsubsection{Utility}

% TODO 177-191

\subsubsection{Describing and Enabling the Invention}

% TODO 35 usc 112

% TODO 195-201 [incandescent lamp]

% TODO 205-226
\newpage % TODO remove
\subsubsection{Novelty and Statutory Bars}

\paragraph{Novelty}

\begin{enumerate}
    \item Old law (1952 Act): defines novelty from the \emph{date of first 
    invention}.
    \item New law (AIA, 2011): novelty is measured as of the \emph{date of the 
    patent application}.\footnote{Casebook p. 26.}
    \item \textbf{Novelty}: new compared to the prior art.
    \item \textbf{Statutory bar}: a bar based on too long a delay in seeking 
    patent protection.
\end{enumerate}

\paragraph{35 U.S.C. \S\ 102 (1952): Conditions for patentability; novelty and 
loss of right to patent}

\begin{enumerate}
    \item (a) Novelty.
    \item (b) Statutory Bars.
    \item (c), (d): (omitted).
    \item (e) Secret Prior Art: Previously filed applications.
    \item (f) Derivation.
    \item (g) (inteference proceedings).
\end{enumerate}

\paragraph{Novelty: \emph{Rosaire v. National Lead Co.}}
~\\\\Earlier use, even if not publicly known, establishes prior art.

\begin{enumerate}
    \item 1936: Rosaire and Horovitz patented a method for analyzing soil to 
    detect nearby oil.\footnote{Casebook p. 228.}
    \item National Lead argued that the patent was invalid and that there had 
    been no infringement.
    \item Rosaire admitted that Teplitz first invented the method, but argued 
    that Teplitz did not publish his ideas and that they were experimental.
    \item Held: Teplitz use was actual, not experimental, and that publication 
    was not required as long as Teplitz did the work openly.
\end{enumerate}

\paragraph{\emph{Inherency Doctrine}}

% FIXME 232-234

\paragraph{Statutory Bars---Defining ``Publication'': \emph{In Re Hall}}
~\\\\Publication of a single copy in a foreign university library can satisfy 
the publication limitation.

\begin{itemize}
    \item Hall applied for a chemical patent. The PTO rejected it because the 
    same process had been published in a German doctoral dissertation more than 
    a year earlier.\footnote{Casebook p. 234.}
    \item On appeal, the court held that publication in a university library was 
    ``sufficiently accessible, at least to the public interested in the art, so 
    that such a one by examining the reference could make the claimed invention 
    without further research or experimentation.''\footnote{Casebook p. 235.} 
    Affirmed.
\end{itemize}

\paragraph{Statutory Bars---Public Use: \emph{Egbert v. Lippmann}}
~\\\\
Hidden use can be public use.

\begin{enumerate}
    \item Barnes invented a new type of corset spring in 1855 but did not apply 
    for a patent until 1866. In the meantime, he had given at least two pairs to 
    Frances, who later became his wife and the assignee of the 
    patent.\footnote{Casebook pp. 237--39.}
    \item Frances sued for patent infringement. The question was whether the 
    invention had been in public use for more than two years\footnote{Now the 
    period is one year.} before the patent application.
    \item The Court (Justice Woods) held that it was a public use because (1) a 
    single use is enough, (2) giving the invention to one person is enough, and 
    (3) selling a hidden component of a public machine is still public use.
    \item Justice Miller, dissenting: if a hidden corset spring is a public use, 
    how can anything be a private use?\footnote{Casebook p. 240.}
\end{enumerate}

\paragraph{Experimental Use Exception: \emph{City of Elizabeth v. Pavement 
Company}}
~\\\\
Experimental use is not public use.

\begin{enumerate}
    \item Nicholson was issued a patent for wooden pavement in 1854. He had been 
    testing the design for the previous six years on a public street in 
    Boston.\footnote{Casebook pp. 243--44.}
    \item Nicholson sued the City of Elizabeth for infringement. Elizabeth 
    argued that the patent was invalid because the invention was in public use.
    \item The court held that experimental use is an exception to the public use 
    restriction. Here, Nicholson intended the use to be experimental, and the 
    public's use was only ``incidental.''\footnote{Casebook pp. 245--46.}
\end{enumerate}
\newpage % TODO remove

% TODO 248-52 priority; novelty and loss of right; griffith v kanamaru

% TODO new AIA sec 102

% TODO 249-257

\subsubsection{Nonobviousness}

% TODO 35 usc 103

% TODO 257-78, 289

\subsection{Administrative Procedures at the PTO}

% TODO 290-95

\subsection{Infringement}

\subsubsection{Claim Interpretation}

% TODO 290-322

\subsubsection{Literal Infringement}

% TODO 323-328

\subsubsection{The Doctrine of Equivalents}

% TODO 328-345

% TODO 350-355

\subsubsection{The ``Reverse'' Doctrine of Equivalents}

% TODO 355-61

\subsection{Defenses}

\subsubsection{Inequitable Conduct}

% TODO 375-82

\subsubsection{Exhaustion of Patent Rights}

% TODO 382-84

% TODO [bowman supp 12-16]

\subsection{Remedies}

% TODO 35 usc 285, 286, 287

% TODO 399-421
