\section{Copyright}

\subsection{Introduction}

\subsubsection{17 U.S.C. \S\ 101: Definitions}

See supplement p. 248.

\subsubsection{History}

\begin{enumerate}
    \item Venice in the fifteenth century recognized exclusive rights in the 
    printing of particular books (as opposed to the technology of creating the 
    books).\footnote{Casebook p. 430.}
    \item \textbf{Statute of Anne} (1710): authors gain exclusive rights for 
    14 years, renewable for another 14 years, subject to registration, notice, 
    and deposit. The government could also set maximum 
    prices.\footnote{Casebook p. 431.}
    \item The continental approach emphasized moral rights, as opposed to the 
    English system of property rights.
    \item The original Copyright Act (1790) mirrored the Statute of Anne. By 
    the end of the nineteenth century, it had been amended to include a range 
    of new technologies (e.g., photography).\footnote{Casebook p. 432.}
    \item \textbf{1909 Act}: extended protection to ``all writings''; extended 
    term to 28 years plus a 28 year renewal.
    \item \textbf{1976 Act}: extended protection to anything ``fixed in a 
    tangible medium of expression''; extended term to life of the author plus 
    50 years; loosened notice and registration requirements; established 
    compulsory licensing regimes; and codified exceptions, including fair use. 
    The 1998 Act extended the term to life of the author plus seventy 
    years.\footnote{Casebook p. 433.}
    \item 1989: the United States joined the Berne convention, scaling back 
    formalities and restoring copyright for foreign works still under 
    protection in the source country.
    \item \textbf{DMCA} (1998): anticircumvention provisions and liability 
    protections for service providers for infringing acts of their 
    subscribers.\footnote{Casebook pp. 433--34.}
\end{enumerate}

\subsubsection{Overview}

% FIXME 434-36

\subsubsection{Philosophy}

% FIXME 436-38

\subsection{Requirements}

\subsubsection{Original Works of Authorship}

% FIXME 438-40

\paragraph{Feist Publications v. Rural Telephone Services}

% FIXME 440-49

\subsubsection{Fixation in a Tangible Medium of Expression}

% TODO 449-454

\subsubsection{Formalities}

% TODO 454-459

\subsection{Copyrightable Subject Matter}

% TODO 17 usc 102(b) [?]

\subsubsection{Limitations on Copyrightability: Distinguishing Function and 
Expression}

% TODO 461-479
% TODO 482-97

\subsubsection{The Domain and Scope of Copyright Protection}

% TODO 497-506

\subsection{Ownership and Duration}

% TODO 17 usc 201c, 301-05

\subsubsection{Initial Ownership of Copyrights}

% TODO 506-526

\subsubsection{Duration and Renewal}

% TODO 526-31

\subsubsection{Division, Transfer, and Reclaiming of Copyrights}

% TODO 531-36

\subsection{Traditional Rights of Copyright Owners}

% TODO 17 usc 106

\subsubsection{The Right to Make Copies}

% TODO 537-73

\subsubsection{The Right to Prepare Derivative Works}

% TODO 573-83

\subsubsection{The Distribution Right}

% TODO 583-88

\subsubsection{Public Performance and Display Rights}

% TODO 588-94

\subsubsection{Moral Rights}

% TODO 594-98

\subsection{Indirect Liability}

% TODO 598-609

\subsection{Defenses}

% TODO 35 usc 107

\subsubsection{Fair Use}

% TODO 623-659

% TODO 670-682

\subsection{Digital Copyright Law}

\subsubsection{Digital Copyright Legislation}

% TODO 690-710

\subsubsection{Fair Use in Cyberspace}

% TODO 732-741

\subsection{International Copyright Law}

\subsubsection{Evolution of the International Copyright System and U.S. 
Participation}

% TODO 742-44

\subsubsection{International Copyright Treaties}

% TODO 744-47

\subsubsection{Protection of U.S. Works Against Infringement Abroad}

% TODO 747-51

\subsubsection{Protection of Foreign Works Against Infringement in the United 
States}

% TODO 751

\subsection{Enforcement and Remedies}

\subsubsection{Injunctions}

% TODO 751-55

\subsubsection{Damages}

% TODO 755-761

\subsubsection{Attorney Fees}

% TODO 761
