\section{State Law and Federal Preemption}

\begin{enumerate}
    \item Six ways that state law protects IP (other than trade secret and 
    trademark:\footnote{Casebook p. 1005.}
    \begin{enumerate}
        \item Misappropriation.
        \item Contract.
        \item Idea submissions (implied contracts).
        \item Right of publicity.
        \item Trespass to chattel.
    \end{enumerate}
\end{enumerate}

\subsection{The Tort of Misappropriation}

\subsubsection{Quasi-Property: \emph{International News Service v. Associated 
Press}}

News stories can be protect as a form of ``quasi-property'' which can be 
protected against misappropriation by direct competitors.

\begin{enumerate}
    \item INS and AP competed to deliver ``hot news'' to newspapers across the 
    country. INS began copying AP's stories and sending them to newspapers 
    itself---for instance, by copying stores published on the East Coast and 
    sending them to West Coast newspapers.\footnote{Casebook p. 1006--07.}
    \item Can AP win an injunction?
    \item Justice Pitney:
    \begin{enumerate}
        \item INS argued that AP did not have a property right in its 
        published stories.
        \item News articles are copyrightable, but the underlying facts, apart 
        from their expression, are not.\footnote{Casebook p. 1008.}
        \item The case depends on whether INS's actions counted as unfair 
        competition.
        \item Since news has value in the context of the competition between 
        INS and AP, courts can consider to be a form of 
        ``quasi-property.''\footnote{Casebook p. 1009.} Misappropriation of 
        this form of property is actionable.
        \item Affirmed (held for AP).
    \end{enumerate}
    \item Justice Holmes, concurring:
    \begin{enumerate}
        \item This is the same harm as misrepresentation (i.e., passing off 
        your goods using another's name).\footnote{Casebook p. 1012--13.}
    \end{enumerate}
    \item Justice Brandeis, dissenting:
    \begin{enumerate}
        \item AP has no property right in its published stories. Failure to 
        give credit is not fraud.\footnote{Casebook p. 1014.}
        \item INS's conduct was unfair, but it's not the court's rule to 
        create a new, complex property right.
    \end{enumerate}
\end{enumerate}

\subsection{Protection by Contract}

\subsubsection{Shrinkwrap Licenses: \emph{ProCD, Inc. v. Zeidenberg}}
~\\\\
Software shrinkwrap license agreements are enforceable, even if the consumer 
can't read them until he has bought the product.

\begin{enumerate}
    \item ProCD published a CD compilation of 3,000 telephone directories. It 
    sold to commercial users at a higher price than individual consumers, and 
    used a shrinkwrap agreement (i.e., EULA) to enforce its price 
    discrimination scheme. Users did not have to agree to the license terms 
    before they bought the product or opened the box, but they did have to 
    agree before using the software.\footnote{Casebook p. 1021--1023.}
    \item Zeidenberg copied ProCD's products and put them online at a lower 
    price.
    \item Zeidenberg argued that only the terms on the outside of the box 
    counted as license terms.
    \item There are plenty of ``money now, terms later'' transactions---like 
    airline tickets, insurance, or concert tickets. Buyers can return the 
    product if they don't agree with the terms. The UCC supports this 
    approach.\footnote{Casebook p. 1024--26..}
    \item Held: the agreement was enforceable.
\end{enumerate}

\subsection{Idea Submissions}

% TODO 1044-64

\subsection{The Right of Publicity}

% TODO 1064-70, 1080-96

\subsection{Patent Preemption of State Laws}

% TODO 1109-18
